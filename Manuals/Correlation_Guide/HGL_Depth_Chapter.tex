% !TEX root = Correlation_Guide.tex

\chapter{Hot Gas Layer Depth}
\label{HGL_Depth_Chapter}

The HGL depth is defined as the distance between the ceiling and the HGL height.

\section{ASET Method}
\label{sec:ASET}

\subsection*{Description}

For a compartment with no ventilation (closed doors) and constant HRR, the available safe egress time (ASET)~\cite{Walton:1}
correlation predicts that the HGL height, $z$~(\si{m}), is given by~\cite{SFPE:Milke}
\be
A\sb{s} \frac{dz}{dt} = \frac{dV\sb{ul}}{dt} = \dot V\sb{ul}
\label{eq:ASET_1}
\ee
where $A\sb{s}$ is the area of the boundary surfaces~(\si{m^2}), and $V\sb{ul}$ is the volume of the HGL~(\si{m^3}).
The change in volume of the upper layer, $\dot V\sb{ul}$~(\si{m^3/s}), is given by
\be
\dot V\sb{ul} = \dot V\sb{exp} + \dot V\sb{ent}
\label{eq:ASET_2}
\ee
The volumetric expansion rate, $\dot V\sb{exp}$~(\si{m^3/s}), is given by~\cite{SFPE:Mowrer}
\be
\dot V\sb{exp} = \frac{\dot Q\sb{net}}{\rho\sb{g} c\sb{p} T\sb{g}} \approx \frac{(1 - \chi\sb{l}) \dot Q\sb{f}}{353} \quad ; \quad \dot Q\sb{net} = \dot Q\sb{f} - \dot Q\sb{l} = \dot Q\sb{f}  (1 - \dot \chi\sb{l})
\label{eq:ASET_3}
\ee
where $\dot Q\sb{net}$, $\dot Q\sb{f}$, and $\dot Q\sb{l}$ are the net HRR, total HRR, and HRR loss to the boundaries~(\si{kW}), respectively, $\rho\sb{g}$, $c\sb{p}$ and $T\sb{g}$ are the density~(\si{kg/m^3}), specific heat~(\si{kJ/(kg.K)}), and temperature~(\si{K}) of air in the HGL, respectively, and $\chi\sb{l}$ is the heat loss fraction to the enclosure boundaries.
The volumetric entrainment rate, $\dot V\sb{ent}$~(\si{m^3/s}), is given by~\cite{Zukoski:1981}
\be
\dot V\sb{ent} = k\sb{v} \dot Q^{1/3} z^{5/3} = \frac{0.21}{K\sb{f}} \left( \frac{g}{\rho_\infty T_\infty} \right)^{1/3} (K\sb{f} \dot Q)^{1/3} (z - z\sb{f})^{5/3}
\label{eq:ASET_4}
\ee
where $k\sb{v}$ is the volumetric entrainment coefficient, $g$ is the acceleration due to gravity~(\si{m/s^2}), $\rho_\infty$ and $T_\infty$ are the density~(\si{kg/m^3}) and temperature~(\si{K}) of ambient air, respectively, $K\sb{f}$ is the location factor, and $z\sb{f}$ is the fuel height~(\si{m}). The location factor has a value of 1, 2, or 4, which corresponds to a fire away from walls or corners, a fire adjacent to a wall, or a fire located in a corner, respectively.

The HGL height, $z$, in Eq.~\ref{eq:ASET_1} can be calculated iteratively using
\be
z|_{t+1} = z|_t - \frac{\dot V\sb{ul}}{L W} \Delta t
\label{eq:ASET_5}
\ee
where $L$ and $W$ are the length and width of the compartment~(\si{m}), respectively, and $\Delta t$ is the time step size~(\si{s}).


\clearpage


\subsection*{Verification}

This example case is based on Test 1 from the NIST and Nuclear Regulatory Commission (NIST/NRC)~\cite{Hamins:SP1013-1} series. This test involved a compartment with a closed door, a heptane spray burner, and no ventilation.

\begin{table}[!ht]
\caption[Verification case, HGL depth]
{Verification case, HGL depth.}
\begin{center}
\begin{tabular}{|l|c|}
\hline
\multicolumn{2}{|c|}{}                                                           \\
\multicolumn{2}{|c|}{User-Specified Input}                                       \\
\multicolumn{2}{|c|}{}                                                           \\ \hline
                            &                                                    \\
\rb{Parameter}              &  \rb{Value}                                        \\ \hline \hline
$\dot Q$ (kW)               &  410                                               \\ \hline
$L$ (m)                     &  21.66                                             \\ \hline
$W$ (m)                     &  7.04                                              \\ \hline
$H$ (m)                     &  3.82                                              \\ \hline
$k$ (\si{kW/(m.K)})         &  0.00012                                           \\ \hline
$\rho$ (kg/m$^3$)           &  737                                               \\ \hline
$c$ (\si{kJ/(kg.K)})        &  1.42                                              \\ \hline
$T_\infty$ (\si{\celsius})  &  22                                                \\ \hline
Location Factor             &  1                                                 \\ \hline
$\chi\sb{l}$                &  0                                                 \\ \hline
$z\sb{f}$                   &  0                                                 \\ \hline
\multicolumn{2}{c}{}                                                             \\ \hline
\multicolumn{2}{|c|}{}                                                           \\
\multicolumn{2}{|c|}{Calculated Output}                                          \\
\multicolumn{2}{|c|}{}                                                           \\ \hline
                                     &                                           \\
\multicolumn{1}{|c|}{\rb{Time}}      &  \multicolumn{1}{c|}{\rb{HGL Depth}}      \\
\multicolumn{1}{|c|}{\rb{(s)}}       &  \multicolumn{1}{c|}{\rb{(m)}}            \\ \hline \hline
\multicolumn{1}{|c|}{0}              &  \multicolumn{1}{c|}{0.00}                \\ \hline
\multicolumn{1}{|c|}{10}             &  \multicolumn{1}{c|}{0.35}                \\ \hline
\multicolumn{1}{|c|}{20}             &  \multicolumn{1}{c|}{0.65}                \\ \hline
\multicolumn{1}{|c|}{30}             &  \multicolumn{1}{c|}{0.93}                \\ \hline
\multicolumn{1}{|c|}{40}             &  \multicolumn{1}{c|}{1.17}                \\ \hline
\multicolumn{1}{|c|}{50}             &  \multicolumn{1}{c|}{1.40}                \\ \hline
\multicolumn{1}{|c|}{60}             &  \multicolumn{1}{c|}{1.60}                \\ \hline
\multicolumn{1}{|c|}{100}            &  \multicolumn{1}{c|}{2.26}                \\ \hline
\multicolumn{1}{|c|}{200}            &  \multicolumn{1}{c|}{3.34}                \\ \hline
\multicolumn{1}{|c|}{300}            &  \multicolumn{1}{c|}{3.82}                \\ \hline
\multicolumn{1}{|c|}{1000}           &  \multicolumn{1}{c|}{3.82}                \\ \hline
\multicolumn{1}{|c|}{1350}           &  \multicolumn{1}{c|}{3.82}                \\ \hline
\end{tabular}
\end{center}
\end{table}


\clearpage


\subsection*{Validation}

A summary of the comparisons between peak predicted and measured HGL depths is shown in Fig.~\ref{HGL_Depth_ASET}.

\begin{figure}[!ht]
\begin{center}
\begin{tabular}{l}
\includegraphics[width=4.0in]{SCRIPT_FIGURES/Scatterplots/HGL_Depth_ASET}
\end{tabular}
\end{center}
\caption[Summary of HGL depth predictions (ASET)]
{Summary of HGL depth predictions using ASET.}
\label{HGL_Depth_ASET}
\end{figure}

This correlation is only valid for closed room tests, and the NIST/NRC tests are the only closed room data set with reduced HGL depth data. Note that the bias and standard deviation are not calculated for these cases because of the limited amount of data.


\clearpage


\section{Yamana and Tanaka Method}
\label{sec:YT}

\subsection*{Description}

For a compartment with no ventilation (closed doors) and constant HRR, the correlation of Yamana and Tanaka~\cite{Tanaka:1} predicts that the HGL height, $z$~(\si{m}), is given by
\be
z = \left( \frac{2 k \dot Q^{1/3} t}{3 A\sb{c}} + \frac{1}{h\sb{c}^{2/3}} \right)^{-3/2}
\label{eq:Yamana_Tanaka}
\ee
where $\dot Q$ is the HRR~(\si{kW}), $t$ is the time after ignition~(\si{s}), $A\sb{c}$ is the compartment floor area~(\si{m^2}), and $h\sb{c}$ is the compartment height~(\si{m}). The constant $k$ is given by
\be
k = \frac{0.076}{(353/T\sb{g})}
\ee
where $T\sb{g}$ is the HGL temperature~(\si{K}).


\clearpage


\subsection*{Verification}

This example case is based on Test 1 from the NIST and Nuclear Regulatory Commission (NIST/NRC)~\cite{Hamins:SP1013-1} series. This test involved a compartment with a closed door, a heptane spray burner, and no ventilation.

Note: In this verification case, the Beyler method (see Section~\ref{sec:Beyler}) is used to calculate the HGL temperature, $T\sb{g}$.

% In the FDTs spreadsheets, this HGL depth correlation is used with the MQH correlation (natural ventilation).

\begin{table}[!ht]
\caption[Verification case, HGL depth]
{Verification case, HGL depth.}
\begin{center}
\begin{tabular}{|l|c|}
\hline
\multicolumn{2}{|c|}{}                                                   \\
\multicolumn{2}{|c|}{User-Specified Input}                               \\
\multicolumn{2}{|c|}{}                                                   \\ \hline
                            &                                            \\
\rb{Parameter}              &  \rb{Value}                                \\ \hline \hline
$\dot Q$ (kW)               &  410                                       \\ \hline
$L$ (m)                     &  21.66                                     \\ \hline
$W$ (m)                     &  7.04                                      \\ \hline
$H$ (m)                     &  3.82                                      \\ \hline
$k$ (\si{kW/(m.K)})         &  0.00012                                   \\ \hline
$\rho$ (kg/m$^3$)           &  737                                       \\ \hline
$c$ (\si{kJ/(kg.K)})        &  1.42                                      \\ \hline
$T_\infty$ (\si{\celsius})  &  22                                        \\ \hline
\multicolumn{2}{c}{}                                                     \\ \hline
\multicolumn{2}{|c|}{}                                                   \\
\multicolumn{2}{|c|}{Calculated Output}                                  \\
\multicolumn{2}{|c|}{}                                                   \\ \hline
                                 &                                       \\
\multicolumn{1}{|c|}{\rb{Time}}  &  \multicolumn{1}{c|}{\rb{HGL Depth}}  \\
\multicolumn{1}{|c|}{\rb{(s)}}   &  \multicolumn{1}{c|}{\rb{(m)}}        \\ \hline \hline
\multicolumn{1}{|c|}{0}          &  \multicolumn{1}{c|}{0.00}            \\ \hline
\multicolumn{1}{|c|}{10}         &  \multicolumn{1}{c|}{0.28}            \\ \hline
\multicolumn{1}{|c|}{20}         &  \multicolumn{1}{c|}{0.53}            \\ \hline
\multicolumn{1}{|c|}{30}         &  \multicolumn{1}{c|}{0.75}            \\ \hline
\multicolumn{1}{|c|}{40}         &  \multicolumn{1}{c|}{0.96}            \\ \hline
\multicolumn{1}{|c|}{50}         &  \multicolumn{1}{c|}{1.15}            \\ \hline
\multicolumn{1}{|c|}{60}         &  \multicolumn{1}{c|}{1.31}            \\ \hline
\multicolumn{1}{|c|}{100}        &  \multicolumn{1}{c|}{1.86}            \\ \hline
\multicolumn{1}{|c|}{200}        &  \multicolumn{1}{c|}{2.64}            \\ \hline
\multicolumn{1}{|c|}{300}        &  \multicolumn{1}{c|}{3.04}            \\ \hline
\multicolumn{1}{|c|}{1000}       &  \multicolumn{1}{c|}{3.67}            \\ \hline
\multicolumn{1}{|c|}{1350}       &  \multicolumn{1}{c|}{3.73}            \\ \hline
\end{tabular}
\end{center}
\end{table}


\clearpage


\subsection*{Validation}

A summary of the comparisons between peak predicted and measured HGL depths is shown in Fig.~\ref{HGL_Depth_YT}.

\begin{figure}[!ht]
\begin{center}
\begin{tabular}{l}
\includegraphics[width=4.0in]{SCRIPT_FIGURES/Scatterplots/HGL_Depth_Yamana_Tanaka}
\end{tabular}
\end{center}
\caption[Summary of HGL depth predictions (Yamana and Tanaka)]
{Summary of HGL depth predictions using Yamana and Tanaka method.}
\label{HGL_Depth_YT}
\end{figure}

This correlation is only valid for closed room tests, and the NIST/NRC tests are the only closed room data set with reduced HGL depth data. Note that the bias and standard deviation are not calculated for these cases because of the limited amount of data.

