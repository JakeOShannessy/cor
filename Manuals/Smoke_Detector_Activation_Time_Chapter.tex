% !TEX root = Correlation_Guide.tex

\chapter{Smoke Detector Activation Time}
\label{Smoke_Detector_Activation_Time_Chapter}

Smoke detector activation can be modeled in a variety of ways. A common method is to assume that the detector behaves like a very sensitive sprinkler with a low activation temperature and RTI.

\subsection*{Description}

For this method, the prediction of smoke detector activation time is identical to that for a sprinkler (as described in Chapter~\ref{Sprinkler_Activation_Time_Chapter}). Equations~\ref{eq:Alpert}, \ref{eq:sprinkler_Tjet}, and \ref{eq:sprinkler_ujet} are used to calculate the time at which the detector reaches its activation temperature, and the input parameters are the same as described in Chapter~\ref{Sprinkler_Activation_Time_Chapter}. Bukowski and Averill~\cite{Bukowski:2} suggested an activation temperature that corresponds to a temperature rise above ambient, $\Delta T_c$, of \SI{5}{\celsius} to be typical of many residential smoke alarms. It is assumed that the smoke detectors are low-RTI devices ($\textrm{RTI}=5~\si{(m.s)^{1/2}}$).

Note that some of these cases assume a quasi-steady approach for a fire source $\dot Q$ that follows a specified $t$-squared growth rate, which was specified as $\dot Q = \alpha t^2$ up to a cutoff time of $t\sb{fire}$. In this approach, for a given time and HRR, Eq.~\ref{eq:Alpert} was used to calculate the time that the detector would activate. If the calculated activation time was less than the current time, then the detector was assumed to activate. After the time $t\sb{fire}$, the fire HRR was steady.


\clearpage


\subsection*{Verification}

This example case is based on Test SDC02 from the NIST Smoke Alarms~\cite{Bukowski:1} series. This test involved ionization and photoelectric smoke alarms located in a single-story manufactured home with a closed door, an upholstered chair fuel source, and no ventilation. If the fire size is not sufficiently large enough to activate the detector, then the activation time is denoted as not applicable (N/A).

\begin{table}[!ht]
\caption[Verification case, smoke detector activation time]
{Verification case, smoke detector activation time.}
\begin{center}
\begin{tabular}{|c|c|c|}
\hline
\multicolumn{3}{|c|}{}                                                                      \\
\multicolumn{3}{|c|}{User-Specified Input}                                                  \\
\multicolumn{3}{|c|}{}                                                                      \\ \hline
\multicolumn{2}{|c|}{}                                  &  \multicolumn{1}{c|}{}            \\
\multicolumn{2}{|l|}{\rb{Parameter}}                    &  \multicolumn{1}{c|}{\rb{Value}}  \\ \hline \hline
\multicolumn{2}{|l|}{$\alpha$ (kW/s$^2$)}               &  \multicolumn{1}{c|}{0.00463}     \\ \hline
\multicolumn{2}{|l|}{Location Factor}                   &  \multicolumn{1}{c|}{1}           \\ \hline
\multicolumn{2}{|l|}{RTI (\si{(m.s)^{1/2}})}            &  \multicolumn{1}{c|}{5}           \\ \hline
\multicolumn{2}{|l|}{$\Delta T\sb{c}$ (\si{\celsius})}  &  \multicolumn{1}{c|}{5}           \\ \hline
\multicolumn{2}{|l|}{$r$ (m)}                           &  \multicolumn{1}{c|}{1.3}         \\ \hline
\multicolumn{2}{|l|}{$H$ (m)}                           &  \multicolumn{1}{c|}{2.1}         \\ \hline
\multicolumn{2}{|l|}{$t\sb{fire}$ (s)}                  &  \multicolumn{1}{c|}{300}         \\ \hline
\multicolumn{2}{|l|}{$T_\infty$ (\si{\celsius})}        &  \multicolumn{1}{c|}{21}          \\ \hline
\multicolumn{2}{c}{}                                                                        \\ \hline
\multicolumn{3}{|c|}{}                                                                      \\
\multicolumn{3}{|c|}{Calculated Output}                                                     \\
\multicolumn{3}{|c|}{}                                                                      \\ \hline
           &             &                                                                  \\
\rb{Time}  &  \rb{HRR}   &  \rb{Activation Time}                                            \\
\rb{(s)}   &  \rb{(kW)}  &  \rb{(s)}                                                        \\ \hline \hline
25         &  2.90       &  N/A                                                             \\ \hline
26         &  3.10       &  N/A                                                             \\ \hline
27         &  3.40       &  N/A                                                             \\ \hline
28         &  3.63       &  35.0                                                            \\ \hline
29         &  3.89       &  23.3                                                            \\ \hline
\end{tabular}
\end{center}
\end{table}

% Note that the FDTs spreadsheet uses incorrect coefficients in Alpert's ceiling jet correlation, which gives an activation time of 35.12 s.


\clearpage


\subsection*{Validation}

A summary of the comparisons between predicted and measured smoke detector activation times is shown in Fig.~\ref{Smoke Detector Activation Time (Temperature Rise)}.

\begin{figure}[!ht]
\begin{center}
\begin{tabular}{l}
\includegraphics[width=4.0in]{SCRIPT_FIGURES/Scatterplots/Smoke_Detector_Activation_Time_Temperature_Rise}
\end{tabular}
\end{center}
\caption[Summary of smoke detector activation time predictions (Temperature Rise)]
{Summary of smoke detector activation time predictions using the Temperature Rise method.}
\label{Smoke Detector Activation Time (Temperature Rise)}
\end{figure}

