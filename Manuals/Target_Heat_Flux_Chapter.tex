% !TEX root = Correlation_Guide.tex

\chapter{Target Heat Flux}
\label{Target_Heat_Flux_Chapter}

Thermal radiation is an important mode of heat transfer in fires. The empirical correlations include simple estimates of flame radiation from a point or cylindrical source.

\section{Point Source Radiation Heat Flux}
\label{sec:Point_Source_Radiation}

\subsection*{Description}

The point source model assumes that radiative energy is concentrated at a point located within a flame~\cite{Beyler2:SFPE}.
Here, the point source is located at a point one-third the height of the flame.
The radiative heat flux, $\dot q''\sb{r}$~(\si{kW/m^2}), at any distance $R$ (\si{m}) from this point can be predicted using
\be
\dot q''\sb{r} = \cos\theta \left( \frac{\chi\sb{r} \dot Q}{4 \pi R^2} \right)
\label{eq:point_source}
\ee
where the $\cos\theta$ term (equal to $x/R$ for targets facing sideways, or $z/R$ for gauges facing upward or downward) accounts for a target that is at an angle $\theta$ from the source. In the Fortran implementation of this empirical correlation, the IOR orientation parameter is used to specify which direction the target or gauge is facing: 1 or -1 for the positive or negative $x$ direction, 2 or -2 for the positive or negative $y$ direction, and 3 or -3 for the positive or negative $z$ direction. In Eq.~\ref{eq:point_source}, $\chi\sb{r}$ is the radiative fraction (unless provided in the test report, a value of 0.35 was used), $\dot Q$ is the HRR of the fire~(\si{kW}), and $R$ is the radial distance from the point source to the edge of the target~(\si{m}) and is given by
\be
R = \sqrt{x^2 + \left(z - \frac{L\sb{f}}{3} \right)^2}
\label{eq:point_source_R}
\ee
where $x$ is the horizontal distance from the point source to the edge of the target~(\si{m}), and $z$ is the height of the heat flux target~(\si{m}). The flame height, $L\sb{f}$ (\si{m}), is given by
\be
L\sb{f} = D (3.7 Q^{*^{2/5}} - 1.02)
\label{eq:point_source_Lf}
\ee
where $D$ is the diameter of the fire source~(\si{m}) and is given by
\be
D = \sqrt{\frac{4 A}{\pi}}
\label{eq:point_source_D}
\ee
where $A$ is the area of the fire source~(m$^2$). The nondimensional HRR, $Q^*$, is given by
\be
Q^* = \frac{\dot Q}{\rho_\infty c\sb{p} T_\infty \sqrt{g} D^{5/2}}
\label{eq:point_source_Qstar}
\ee
where $\rho_\infty$ is the ambient air density~(\si{kg/m^3}), $c\sb{p}$ is the specific heat of air~(\si{kJ/(kg.K)}), $T_\infty$ is the ambient air temperature~(\si{K}), and $g$ is the acceleration of gravity~(\si{m/s^2}).

\subsection*{Verification}

This example case is based on Test 7 from the Fleury Heat Flux~\cite{Fleury:Masters} series. This test involved a \SI{100}{kW} propane burner with dimensions of 0.3~m by 0.3~m and heat flux measurements at various distances from the burner.

% In the FDTs spreadsheets, the point source radiation heat flux is calculated without accounting for the angle/elevation of the target with respect to the fire, and the flame height is not considered. The correlations.f90 program considers both of these and assumes that the point source is located one-third of the height of the flame ($L\sb{f}/3$).

\begin{table}[!ht]
\caption[Verification case, point source radiation heat flux]
{Verification case, point source radiation heat flux.}
\begin{center}
\begin{tabular}{|l|l|}
\hline
\multicolumn{2}{|c|}{}                                                      \\
\multicolumn{2}{|c|}{User-Specified Input}                                  \\
\multicolumn{2}{|c|}{}                                                      \\ \hline
                                   &                                        \\
\rb{Parameter}                     &  \rb{Value}                            \\ \hline \hline
$\dot Q$ (kW)                      &  100                                   \\ \hline
$\chi\sb{r}$                       &  0.35                                  \\ \hline
$A$ (m$^2$)                        &  0.09                                  \\ \hline
$x$ (m)                            &  0.50, 0.75, 1.0                       \\ \hline
$z$ (m)                            &  0.0                                   \\ \hline
IOR                                &  2                                     \\ \hline
\multicolumn{2}{c}{}                                                        \\ \hline
\multicolumn{2}{|c|}{}                                                      \\
\multicolumn{2}{|c|}{Calculated Output}                                     \\
\multicolumn{2}{|c|}{}                                                      \\ \hline
                                   &                                        \\
\multicolumn{1}{|c|}{\rb{Radius}}  &  \multicolumn{1}{c|}{\rb{Heat Flux}}   \\
\multicolumn{1}{|c|}{\rb{(m)}}     &  \multicolumn{1}{c|}{\rb{(kW/m$^2$)}}  \\ \hline \hline
\multicolumn{1}{|c|}{0.50}         &  \multicolumn{1}{c|}{6.01}             \\ \hline
\multicolumn{1}{|c|}{0.75}         &  \multicolumn{1}{c|}{3.65}             \\ \hline
\multicolumn{1}{|c|}{1.00}         &  \multicolumn{1}{c|}{2.33}             \\ \hline
\end{tabular}
\end{center}
\end{table}

% \noindent Expected result (FDTs spreadsheets): At a radius $R$ of 0.67~m, the radiative heat flux $\dot q''\sb{r}$ is 6.22~kW/m$^2$; at an $R$ of 0.92~m, $\dot q''\sb{r}$ is 3.30~kW/m$^2$; at an $R$ of 1.17~m, $\dot q''\sb{r}$ is 2.04~kW/m$^2$.


\clearpage


\subsection*{Validation}

A summary of the comparisons between peak predicted and measured heat fluxes is shown in Fig.~\ref{Target Heat Flux (Point Source)}.

\begin{figure}[!ht]
\begin{center}
\begin{tabular}{l}
\includegraphics[width=4.0in]{SCRIPT_FIGURES/Scatterplots/Target_Heat_Flux_PS}
\end{tabular}
\end{center}
\caption[Summary of point source radiation heat flux predictions]
{Summary of point source radiation heat flux predictions.}
\label{Target Heat Flux (Point Source)}
\end{figure}


\clearpage


\section{Solid Flame Radiation Heat Flux}
\label{sec:Solid_Flame_Radiation}

\subsection*{Description}

The solid flame model predicts the heat flux to a target based on the effective emissive power from a flame and a view factor calculation~\cite{Beyler2:SFPE}.
The radiative heat flux, $\dot q''\sb{r}$~(\si{kW/m^2}), at any distance $R$ (\si{m}) from the center of the flame can be predicted using
\be
\dot q''\sb{r} = E F_{12}
\label{eq:solid_flame}
\ee
where $E$ is the effective emissive power of the flame~(\si{kW/m^2}), and $F_{12}$ is the view factor between the target and the flame.
The effective emissive power of the flame, $E$, is given by
\be
E = 58 (10^{-0.0823 D})
\label{eq:solid_flame_E}
\ee
where $D$ is the effective fire diameter~(\si{m}).

\subsection*{Ground-Level Target}

\noindent For a heat flux target that is at the same level as the base of the flame, the view factor between the target and the flame, $F_{12}$, is given by
\be
F_{12} = \sqrt{F\sb{12, H}^2 + F\sb{12, V}^2}
\label{eq:solid_flame_F12_HV}
\ee
where $F\sb{12, H}$ and $F\sb{12, V}$ are given by
\begin{align}
F\sb{12, H} &= \frac{\left( B - \frac{1}{S} \right)}{\pi \sqrt{B^2 - 1}} \tan^{-1} \sqrt{ \frac{(B+1)(S-1)}{(B-1)(S+1)} } -
\frac{\left( A - \frac{1}{S} \right)}{\pi \sqrt{A^2 - 1}} \tan^{-1} \sqrt{ \frac{(A+1)(S-1)}{(A-1)(S+1)} } \\
F\sb{12, V} &= \frac{1}{\pi S} \tan^{-1} \left( \frac{H}{\sqrt{S^2-1}} \right) - \frac{H}{\pi S} \tan^{-1} \sqrt{\frac{S-1}{S+1}} +
\frac{A H}{\pi S \sqrt{A^2-1}} \tan^{-1} \sqrt{\frac{(A+1)(S-1)}{(A-1)(S+1)}}
\label{eq:solid_flame_F12_H_V}
\end{align}
where $H$, $S$, $A$, and $B$ are given by
\begin{align}
H &= \frac{2 L\sb{f}}{D}        \\
S &= \frac{2 x}{D}              \\
A &= \frac{H^2 + S^2 + 1}{2 S}  \\
B &= \frac{1+S^2}{2 S}
\label{eq:solid_flame_H_S_A_B}
\end{align}
where $x$ is the horizontal distance from the center of the flame to the edge of the heat
flux target~(\si{m}), and $L\sb{f}$ is the flame height given in Eq.~\ref{eq:point_source_Lf}~(\si{m}).


\clearpage


\subsection*{Elevated Target}

\noindent For a heat flux target that is elevated from the base of the flame, the view factor between the target and the flame, $F_{12}$, is given by
\be
F_{12} = F\sb{12, V1} + F\sb{12, V2}
\label{eq:solid_flame_F12_V1V2}
\ee
where $F\sb{12, V1}$ and $F\sb{12, V2}$ are given by
\begin{align}
F\sb{12, V1} &= \frac{1}{\pi S} \tan^{-1} \left( \frac{H_1}{\sqrt{S^2-1}} \right) - \frac{H_1}{\pi S} \tan^{-1} \sqrt{\frac{S-1}{S+1}} +
\frac{A_1 H_1}{\pi S \sqrt{A_1^2-1}} \tan^{-1} \sqrt{\frac{(A_1+1)(S-1)}{(A_1-1)(S+1)}} \\
F\sb{12, V2} &= \frac{1}{\pi S} \tan^{-1} \left( \frac{H_2}{\sqrt{S^2-1}} \right) - \frac{H_2}{\pi S} \tan^{-1} \sqrt{\frac{S-1}{S+1}} +
\frac{A_2 H_2}{\pi S \sqrt{A_2^2-1}} \tan^{-1} \sqrt{\frac{(A_2+1)(S-1)}{(A_2-1)(S+1)}}
\label{eq:solid_flame_F12_V1_V2}
\end{align}
where $H_1$, $H_2$, $S$, $A_1$, and $A_2$ are given by
\begin{align}
H_1 &= \frac{2 z}{D}                \\
H_2 &= \frac{2 (L\sb{f}-z)}{D}      \\
S   &= \frac{2 x}{D}                \\
A_1 &= \frac{H_1^2 + S^2 + 1}{2 S}  \\
A_2 &= \frac{H_2^2 + S^2 + 1}{2 S}
\label{eq:solid_flame_H_S_A}
\end{align}
where $x$ is the horizontal distance from the center of the flame to the edge of the heat
flux target~(\si{m}), $z$ is the height of the target from the base of the flame~(\si{m}),
and $L\sb{f}$ is the flame height given in Eq.~\ref{eq:point_source_Lf}~(\si{m}).


\clearpage


\subsection*{Verification}

This example case is based on Test 7 from the Fleury Heat Flux~\cite{Fleury:Masters} series. This test involved a \SI{100}{kW} propane burner with dimensions of 0.3~m by 0.3~m and heat flux measurements at various distances from the burner.

\begin{table}[!ht]
\caption[Verification case, solid flame radiation heat flux]
{Verification case, solid flame radiation heat flux.}
\begin{center}
\begin{tabular}{|l|l|}
\hline
\multicolumn{2}{|c|}{}                                                      \\
\multicolumn{2}{|c|}{User-Specified Input}                                  \\
\multicolumn{2}{|c|}{}                                                      \\ \hline
                &                                                           \\
\rb{Parameter}  &  \rb{Value}                                               \\ \hline \hline
$\dot Q$ (kW)   &  100                                                      \\ \hline
$A$ (m$^2$)     &  0.09                                                     \\ \hline
$x$ (m)         &  0.50, 0.75, 1.0                                          \\ \hline
$z$ (m)         &  0.0                                                      \\ \hline
\multicolumn{2}{c}{}                                                        \\ \hline
\multicolumn{2}{|c|}{}                                                      \\
\multicolumn{2}{|c|}{Calculated Output}                                     \\
\multicolumn{2}{|c|}{}                                                      \\ \hline
                                   &                                        \\
\multicolumn{1}{|c|}{\rb{Radius}}  &  \multicolumn{1}{c|}{\rb{Heat Flux}}   \\
\multicolumn{1}{|c|}{\rb{(m)}}     &  \multicolumn{1}{c|}{\rb{(kW/m$^2$)}}  \\ \hline \hline
\multicolumn{1}{|c|}{0.50}         &  \multicolumn{1}{c|}{11.18}            \\ \hline
\multicolumn{1}{|c|}{0.75}         &  \multicolumn{1}{c|}{6.90}             \\ \hline
\multicolumn{1}{|c|}{1.00}         &  \multicolumn{1}{c|}{4.69}             \\ \hline
\end{tabular}
\end{center}
\end{table}


\clearpage


\subsection*{Validation}

A summary of the comparisons between peak predicted and measured plume temperatures is shown in Fig.~\ref{Target Heat Flux (Solid Flame)}.

\begin{figure}[!ht]
\begin{center}
\begin{tabular}{l}
\includegraphics[width=4.0in]{SCRIPT_FIGURES/Scatterplots/Target_Heat_Flux_SF}
\end{tabular}
\end{center}
\caption[Summary of solid flame radiation heat flux predictions]
{Summary of solid flame radiation heat flux predictions.}
\label{Target Heat Flux (Solid Flame)}
\end{figure}
