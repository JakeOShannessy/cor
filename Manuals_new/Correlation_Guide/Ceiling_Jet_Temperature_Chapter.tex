% !TEX root = Correlation_Guide.tex

\chapter{Ceiling Jet Temperature}
\label{Ceiling_Jet_Temperature_Chapter}

The ceiling jet is the shallow layer of hot gases below the ceiling that spreads radially from the centerline of the fire plume. The ceiling jet has a higher temperature than the overall temperature of the HGL, and therefore it is important where targets are located just below the ceiling.

\subsection*{Description}

For a steady-state fire, the correlation of Alpert~\cite{SFPE:Alpert} predicts that the ceiling jet temperature rise, $\Delta T\sb{jet}$~(\si{\celsius}), from a fire plume is given by
\be
\Delta T\sb{jet} = \left\{ \begin{array}{cl}
   \frac{16.9 \dot Q^{2/3}}{H^{5/3}}  &  r/H <= 0.18 \\[0.1in]
   \frac{5.38 (\dot Q / r)^{2/3}}{H}  &  r/H >  0.18 
   \end{array} \right.
\label{eq:Alpert_Tjet}
\ee
where $\dot Q$ is the total HRR~(\si{kW}), $H$ is the height of the ceiling above the fuel~(\si{m}), and $r$~is the radial distance to the detector~(\si{m}).

Note that some of these cases assume a quasi-steady approach for a fire source $\dot Q$ that follows a specified $t$-squared growth rate, which is given by
\be
\dot Q = \alpha t^2
\label{eq:t_squared}
\ee
where $\alpha$ is the $t$-squared growth rate parameter~(\si{kW/s^2}), and $t$ is time~(\si{s}).

For cases in which the fire was located against a wall or corner, these correlations are adjusted based on the method of reflection. For a fire adjacent to a flat wall, 2$\dot Q$ is substituted for $\dot Q$; and for a fire in a 90-degree corner, 4$\dot Q$ is substituted for $\dot Q$. This adjustment is denoted in the input parameters as the location factor. For a given case, the location factor has a value of 1, 2, or 4, which corresponds to a fire away from walls or corners, a fire adjacent to a wall, or a fire located in a corner, respectively.


\clearpage


\subsection*{Verification}

This example case is based on Test 1 from the NIST and Nuclear Regulatory Commission (NIST/NRC)~\cite{Hamins:SP1013-1} series. This test involved a compartment with a closed door, a heptane spray burner, and no ventilation.

\begin{table}[!ht]
\caption[Verification case, ceiling jet temperature]
{Verification case, ceiling jet temperature.}
\begin{center}
\begin{tabular}{|l|c|}
\hline
\multicolumn{2}{|c|}{}                              \\
\multicolumn{2}{|c|}{User-Specified Input}          \\
\multicolumn{2}{|c|}{}                              \\ \hline
                              &                     \\
\rb{Parameter}                &  \rb{Value}         \\ \hline \hline
$\dot Q$ (kW)                 &  410                \\ \hline
Location Factor               &  1                  \\ \hline
$r$ (m)                       &  5.90               \\ \hline
$H$ (m)                       &  3.72               \\ \hline
$T_{\infty}$ (\si{\celsius})  &  22                 \\ \hline
\multicolumn{2}{c}{}                                \\ \hline
\multicolumn{2}{|c|}{}                              \\
\multicolumn{2}{|c|}{Calculated Output}             \\
\multicolumn{2}{|c|}{}                              \\ \hline
\multicolumn{2}{|c|}{}                              \\
\multicolumn{2}{|c|}{\rb{Ceiling Jet Temperature}}  \\
\multicolumn{2}{|c|}{\rb{(\si{\celsius})}}          \\ \hline \hline
\multicolumn{2}{|c|}{46.45}                         \\ \hline
\end{tabular}
\end{center}
\end{table}


\clearpage


\subsection*{Validation}

A summary of the comparisons between peak predicted and measured ceiling jet temperatures is shown in Fig.~\ref{Ceiling Jet Temperature, Unconfined (Alpert)}.

It is important to note that this ceiling jet temperature correlation was developed using data from tests that were conducted in a large facility in which the distant walls and large compartment size did not allow for the development of a significant hot gas layer. In a more typical fire scenario (i.e., a smaller compartment), the HGL develops relatively quickly, and temperatures at the ceiling are affected by the ceiling jet as well as the accumulating HGL. Thus, when compared to experimentally measured ceiling temperatures in a compartment fire, this correlation tends to underpredict the temperatures because it is not accounting for the development of the HGL. This is an important consideration when using this correlation to predict detector or sprinkler activations.

For the reasons stated above, two scatter plot comparisons are shown in Fig.~\ref{Ceiling Jet Temperature, Unconfined (Alpert)}. One scatter plot shows the results for unconfined tests that were conducted under a false ceiling in which the hot plume gases did not accumulate to form an HGL, but were allowed to spill out from under a false ceiling. The other scatter plot shows the results of underpredicted temperature comparisons for compartment fire tests. The use of the ceiling jet correlation in a confined compartment with the presence of an HGL can result in an underprediction of the measured ceiling jet temperature by approximately 70~\%. Therefore, the model bias factor and model relative standard deviation were only calculated for the unconfined ceiling jet cases that the correlation was developed for. In the unconfined ceiling cases, the ceiling temperature predictions are in better agreement with experimental data because this scenario is more representative of a temperature rise due to only ceiling jet flow from the fire plume. 

\begin{figure}[!ht]
\begin{center}
\begin{tabular}{l}
\includegraphics[width=3.9in]{SCRIPT_FIGURES/Scatterplots/Ceiling_Jet_Temperature_Unconfined} \\
\includegraphics[width=3.9in]{SCRIPT_FIGURES/Scatterplots/Ceiling_Jet_Temperature_Compartment}
\end{tabular}
\end{center}
\caption[Summary of ceiling jet temperature predictions]
{Summary of compartment (top) and unconfined (bottom) ceiling jet temperature predictions.}
\label{Ceiling Jet Temperature, Unconfined (Alpert)}
\end{figure}
