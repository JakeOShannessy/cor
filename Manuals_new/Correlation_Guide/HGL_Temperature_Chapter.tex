% !TEX root = Correlation_Guide.tex

\chapter{Hot Gas Layer Temperature}
\label{HGL_Temperature_Chapter}

The empirical correlations can predict an average hot gas layer (HGL) temperature. Because there are different empirical correlations for compartments that are naturally ventilated, mechanically ventilated, or unventilated, the results for HGL temperature are divided into three categories.

\section{Natural Ventilation (MQH)}
\label{sec:MQH}

\subsection*{Description}

For a compartment with natural ventilation, the correlation of McCaffrey, Quintiere, and Harkleroad (MQH)~\cite{SFPE:Walton} predicts that the hot gas layer (HGL) temperature rise, $\Delta T\sb{g}$~(\si{\celsius}), is given by
\be
\Delta T\sb{g} = 6.85 \left( \frac{\dot Q^2}{A\sb{o} \sqrt{H\sb{o}} h\sb{k} A\sb{T}} \right)^{1/3}
\label{eq:MQH}
\ee
where $\dot Q$ is the total heat release rate (HRR) of the fire~(\si{kW}), $A\sb{o}$ is the area of the ventilation opening~(\si{m^2}), $H\sb{o}$ is the height of the ventilation opening~(\si{m}), and $A\sb{T}$ is the total area of the compartment enclosing surfaces~(\si{m^2}), excluding areas of vent openings, which is given by
\be
A_T = 2 L W + 2 L H + 2 W H - H\sb{o} W\sb{o}
\label{eq:A_T}
\ee
where $L$, $W$, $H$ are the length, width, and height of the compartment~(\si{m}), respectively, and $W\sb{o}$ is the width of the ventilation opening~(\si{m}). The heat transfer coefficient, $h\sb{k}$ (\si{kW/(m^2.K})), is given by
\be
h\sb{k} = \left\{ \begin{array}{cl}
   \sqrt{k \rho c/t}  & t \le t\sb{p} \\[0.1in]
   k/\delta           & t > t\sb{p}
   \end{array} \right.
\label{eq:MQH_hk_lt}
\ee
where $k$ is the thermal conductivity of the interior lining~(\si{kW/(m.K)}), $\rho$ is its density~(\si{kg/m^3}), $c$ is its specific heat~(\si{kJ/(kg.K)}), and $\delta$ is its thickness~(\si{m}). The thermal penetration time, $t\sb{p}$ (\si{\second}), is given by
\be
t\sb{p} = \left( \frac{\rho c}{k} \right) \left( \frac{\delta}{2} \right)^2
\label{eq:MQH_tp}
\ee


\clearpage


\subsection*{Verification}

This example case is based on Test 51 from the Lawrence-Livermore National Laboratory (LLNL)~\cite{Foote:LLNL1986} series. This test involved a compartment with an open door, a methane burner, and natural ventilation.

\begin{table}[!ht]
\caption[Verification case, HGL temperature, natural ventilation]
{Verification case, HGL temperature, natural ventilation.}
\begin{center}
\begin{tabular}{|l|c|}
\hline
\multicolumn{2}{|c|}{}                                                         \\
\multicolumn{2}{|c|}{User-Specified Input}                                     \\
\multicolumn{2}{|c|}{}                                                         \\ \hline
                            &                                                  \\
\rb{Parameter}              &  \rb{Value}                                      \\ \hline \hline
$\dot Q$ (kW)               &  200                                             \\ \hline
$L$ (m)                     &  6.0                                             \\ \hline
$W$ (m)                     &  4.0                                             \\ \hline
$H$ (m)                     &  3.0                                             \\ \hline
$H\sb{o}$ (m)               &  2.06                                            \\ \hline
$W\sb{o}$ (m)               &  0.76                                            \\ \hline
$k$ (\si{kW/(m.K)})         &  0.000463                                        \\ \hline
$\rho$ (kg/m$^3$)           &  1607                                            \\ \hline
$c$ (\si{kJ/(kg.K)})        &  1.0                                             \\ \hline
$\delta$ (m)                &  0.10                                            \\ \hline
$T_\infty$ (\si{\celsius})  &  33                                              \\ \hline
\multicolumn{2}{c}{}                                                           \\ \hline
\multicolumn{2}{|c|}{}                                                         \\
\multicolumn{2}{|c|}{Calculated Output}                                        \\
\multicolumn{2}{|c|}{}                                                         \\ \hline
                                 &                                             \\
\multicolumn{1}{|c|}{\rb{Time}}  &  \multicolumn{1}{c|}{\rb{HGL Temperature}}  \\
\multicolumn{1}{|c|}{\rb{(s)}}   &  \multicolumn{1}{c|}{\rb{(\si{\celsius})}}  \\ \hline \hline
\multicolumn{1}{|c|}{0}          &  \multicolumn{1}{c|}{33.00}                 \\ \hline
\multicolumn{1}{|c|}{60}         &  \multicolumn{1}{c|}{111.45}                \\ \hline
\multicolumn{1}{|c|}{120}        &  \multicolumn{1}{c|}{121.05}                \\ \hline
\multicolumn{1}{|c|}{180}        &  \multicolumn{1}{c|}{127.21}                \\ \hline
\multicolumn{1}{|c|}{600}        &  \multicolumn{1}{c|}{148.14}                \\ \hline
\multicolumn{1}{|c|}{1200}       &  \multicolumn{1}{c|}{162.24}                \\ \hline
\multicolumn{1}{|c|}{1800}       &  \multicolumn{1}{c|}{171.28}                \\ \hline
\multicolumn{1}{|c|}{2400}       &  \multicolumn{1}{c|}{178.07}                \\ \hline
\multicolumn{1}{|c|}{3000}       &  \multicolumn{1}{c|}{183.57}                \\ \hline
\end{tabular}
\end{center}
\end{table}

\noindent This verification example serves as both a worked example case and a check on the mathematical implementation of this empirical correlation for software quality assurance (SQA) purposes, hence the extended number of significant digits. The verification examples are similar for all of the remaining empirical correlations in this document.


\clearpage


\subsection*{Validation}

A summary of the comparisons between peak predicted and measured compartment temperatures is shown in Fig.~\ref{HGL Temperature, Natural Ventilation (MQH)}.

\begin{figure}[!ht]
\begin{center}
\begin{tabular}{l}
\includegraphics[width=4.0in]{SCRIPT_FIGURES/Scatterplots/HGL_Temperature_MQH}
\end{tabular}
\end{center}
\caption[Summary of HGL temperature predictions for natural ventilation tests (MQH)]
{Summary of HGL temperature predictions for natural ventilation tests using the MQH method.}
\label{HGL Temperature, Natural Ventilation (MQH)}
\end{figure}


\clearpage


\subsection*{Explanation of Statistical Metrics}

In Fig.~\ref{HGL Temperature, Natural Ventilation (MQH)}, the measured values are represented by the horizontal axis and the predicted values by the vertical axis. If a particular prediction and measurement are the same, then the resulting point falls on the solid diagonal line. To better make use of these results, two statistical parameters are calculated for each empirical correlation and each predicted quantity. The first parameter, $\delta$, is the bias factor, which indicates the extent to which the empirical correlation, on average, under or over-predicts the measurements of a given quantity. It is assumed that the experiments are unbiased; that is, the bias factor for the experimental measurements is 1. For example, a bias factor of 1.02 indicates that the model over-estimates the measured quantity by \SI{2}{\percent}, on average. The bias factor is shown graphically by the solid red line.

The second parameter, $\widetilde{\sigma}\sb{M}$, is the relative standard deviation of the model, which indicates the variability of the model. In Fig.~\ref{HGL Temperature, Natural Ventilation (MQH)}, there are two sets of off-diagonal lines. The first set, shown as dashed black lines, indicate the uncertainty of the experimental measurements in terms of a relative standard deviation, $\widetilde{\sigma}\sb{E}$. The experimental relative standard deviation was determined by considering the systematic and random uncertainty values for each measurement quantity, which is described in more detail in the ``Experimental Uncertainty'' section of the FDS Validation Guide~\cite{FDS_Validation_Guide}. The slopes of the dashed black lines are~$1 \pm 2\widetilde{\sigma}\sb{E}$, which represents the \SI{95}{\percent} confidence intervals. The set of red dashed lines indicate the model relative standard deviation, $\widetilde{\sigma}\sb{M}$. The model relative standard deviation is reported as one standard deviation of the predicted quantity. The slopes of these lines are~$\delta \pm 2\widetilde{\sigma}\sb{M}$. If the model was as accurate as the measurements against which it is compared, then the two sets of off-diagonal lines would merge. The extent to which the data scatters outside of the experimental bounds is an indication of the degree of uncertainty of the empirical correlations.

These symbols and nomenclature are similar for all of the remaining scatter plots in this document. More detailed discussion of the experimental and model relative standard deviations is provided in the FDS Validation Guide~\cite{FDS_Validation_Guide} and in McGrattan and Toman~\cite{McGrattan:Metrologia}.


\clearpage


\section{Forced Ventilation (FPA)}
\label{sec:FPA}

\subsection*{Description}

For a compartment with forced ventilation, the correlation of Foote, Pagni, and Alvares (FPA)~\cite{SFPE:Walton} predicts that the HGL temperature rise, $\Delta T\sb{g}$~(\si{\celsius}), is given by
\be
\Delta T\sb{g} = \left[ 0.63 \left( \frac{\dot Q}{\dot m\sb{g} c\sb{p} T_\infty} \right)^{0.72} \left( \frac{h\sb{k} A\sb{T}}{\dot m\sb{g} c\sb{p}} \right)^{-0.36} \right] T_\infty
\label{eq:FPA}
\ee
where $\dot Q$ is the HRR of the fire~(\si{kW}), $\dot m\sb{g}$ is the compartment ventilation mass flow rate~(\si{kg/s}), $c\sb{p}$ is the specific heat of air (\si{kJ/(kg.K)}), $T_\infty$ is the ambient air temperature~(\si{\celsius}), $h\sb{k}$ is the heat transfer coefficient~(\si{kW/(m^2.K)}), and $A\sb{T}$ is the total area of the compartment enclosing surfaces~(\si{m^2}), excluding areas of vent openings. The heat transfer coefficient, $h\sb{k}$ (\si{kW/(m^2.K)}), is given by
\be
h\sb{k} = \left\{ \begin{array}{cl}
   \sqrt{k \rho c/t}  & t \le t\sb{p} \\[0.1in]
   k/\delta           & t > t\sb{p}
   \end{array} \right.
\label{eq:FPA_hk_lt}
\ee
where $k$ is the thermal conductivity of the interior lining~(\si{kW/(m.K)}), $\rho$ is its density~(\si{kg/m^3}), $c$ is its specific heat~(\si{kJ/(kg.K)}), and $\delta$ is its thickness~(\si{m}). The thermal penetration time, $t\sb{p}$ (\si{\second}), is given by
\be
t\sb{p} = \left( \frac{\rho c}{k} \right) \left( \frac{\delta}{2} \right)^2
\label{eq:FPA_tp}
\ee


\clearpage


\subsection*{Verification}

This example case is based on Test 1 from the Factory Mutual and Sandia National Laboratories (FM/SNL)~\cite{Nowlen:NUREG4681, Nowlen:NUREG4527} series. This test involved a compartment with an open door, a propylene burner, and forced ventilation.

\begin{table}[!ht]
\caption[Verification case, HGL temperature, forced ventilation]
{Verification case, HGL temperature, forced ventilation.}
\begin{center}
\begin{tabular}{|l|c|}
\hline
\multicolumn{2}{|c|}{}                                                         \\
\multicolumn{2}{|c|}{User-Specified Input}                                     \\
\multicolumn{2}{|c|}{}                                                         \\ \hline
                            &                                                  \\
\rb{Parameter}              &  \rb{Value}                                      \\ \hline \hline
$\dot Q$ (kW)               &  516                                             \\ \hline
$\dot m$ (kg/s)             &  4.5                                             \\ \hline
$c\sb{p}$ (\si{kJ/(kg.K)})  &  1.0                                             \\ \hline
$L$ (m)                     &  18.3                                            \\ \hline
$W$ (m)                     &  12.2                                            \\ \hline
$H$ (m)                     &  6.1                                             \\ \hline
$k$ (\si{kW/(m.K)})         &  0.00023                                         \\ \hline
$\rho$ (kg/m$^3$)           &  1000                                            \\ \hline
$c$ (\si{kJ/(kg.K)})        &  1.16                                            \\ \hline
$\delta$ (m)                &  0.025                                           \\ \hline
$T_\infty$ (\si{\celsius})  &  15                                              \\ \hline
\multicolumn{2}{c}{}                                                           \\ \hline
\multicolumn{2}{|c|}{}                                                         \\
\multicolumn{2}{|c|}{Calculated Output}                                        \\ 
\multicolumn{2}{|c|}{}                                                         \\ \hline
                                 &                                             \\
\multicolumn{1}{|c|}{\rb{Time}}  &  \multicolumn{1}{c|}{\rb{HGL Temperature}}  \\
\multicolumn{1}{|c|}{\rb{(s)}}   &  \multicolumn{1}{c|}{\rb{(\si{\celsius})}}  \\ \hline \hline
\multicolumn{1}{|c|}{0}          &  \multicolumn{1}{c|}{15.00}                 \\ \hline
\multicolumn{1}{|c|}{60}         &  \multicolumn{1}{c|}{53.07}                 \\ \hline
\multicolumn{1}{|c|}{120}        &  \multicolumn{1}{c|}{58.13}                 \\ \hline
\multicolumn{1}{|c|}{180}        &  \multicolumn{1}{c|}{61.39}                 \\ \hline
\multicolumn{1}{|c|}{240}        &  \multicolumn{1}{c|}{63.86}                 \\ \hline
\multicolumn{1}{|c|}{300}        &  \multicolumn{1}{c|}{65.86}                 \\ \hline
\multicolumn{1}{|c|}{360}        &  \multicolumn{1}{c|}{67.56}                 \\ \hline
\multicolumn{1}{|c|}{420}        &  \multicolumn{1}{c|}{69.04}                 \\ \hline
\multicolumn{1}{|c|}{480}        &  \multicolumn{1}{c|}{70.35}                 \\ \hline
\multicolumn{1}{|c|}{540}        &  \multicolumn{1}{c|}{71.54}                 \\ \hline
\multicolumn{1}{|c|}{600}        &  \multicolumn{1}{c|}{72.62}                 \\ \hline
\end{tabular}
\end{center}
\end{table}


\clearpage


\subsection*{Validation}

A summary of the comparisons between peak predicted and measured compartment temperatures is shown in Fig.~\ref{HGL Temperature, Forced Ventilation (FPA)}.

\begin{figure}[!ht]
\begin{center}
\includegraphics[width=4.0in]{SCRIPT_FIGURES/Scatterplots/HGL_Temperature_FPA}
\end{center}
\caption[Summary of HGL temperature predictions for forced ventilation tests (FPA)]
{Summary of HGL temperature predictions for forced ventilation tests using the FPA method.}
\label{HGL Temperature, Forced Ventilation (FPA)}
\end{figure}

Note that the LLNL Enclosure experiments were used to develop the FPA correlation.


\clearpage


\section{Forced Ventilation (DB)}
\label{sec:DB}

\subsection*{Description}

For a compartment with forced ventilation, the correlation of Deal and Beyler (DB)~\cite{SFPE:Walton} predicts that the HGL temperature rise, $\Delta T\sb{g}$~(\si{\celsius}), is given by
\be
\Delta T\sb{g} = \left( \frac{\dot Q}{\dot m\sb{g} c\sb{p} + h\sb{k} A\sb{T}} \right)
\label{eq:DB}
\ee
where $\dot Q$ is the HRR of the fire~(\si{kW}), $\dot m\sb{g}$ is the compartment ventilation mass flow rate~(\si{kg/s}), $c\sb{p}$ is the specific heat of air (\si{kJ/(kg.K)}), $T_\infty$ is the ambient air temperature~(\si{\celsius}), $h\sb{k}$ is the heat transfer coefficient~(\si{kW/(m^2.K)}), and $A\sb{T}$ is the total area of compartment enclosing surfaces~(\si{m^2}), excluding areas of vent openings. The heat transfer coefficient, $h\sb{k}$ (\si{kW/(m^2.K)}), is given by
\be
h\sb{k} = 0.4\ \mathrm{max} \left( \sqrt{\frac{k \rho c}{t}} , \frac{k}{\delta} \right)
\label{eq:DB_hk}
\ee
where $k$ is the thermal conductivity of the interior lining~(\si{kW/(m.K)}), $\rho$ is the density of the interior lining~(\si{kg/m^3}), $c$ is the specific heat of the interior lining~(\si{kJ/(kg.K)}), $t$ is the exposure time~(\si{\second}), and $\delta$ is the thickness of the interior lining~(\si{m}). This model is only valid for times up to 2000 seconds.


\clearpage


\subsection*{Verification}

This example case is based on Test 1 from the Factory Mutual and Sandia National Laboratories (FM/SNL)~\cite{Nowlen:NUREG4681, Nowlen:NUREG4527} series. This test involved a compartment with an open door, a propylene burner, and forced ventilation.

\begin{table}[!ht]
\caption[Verification case, HGL temperature, forced ventilation]
{Verification case, HGL temperature, forced ventilation.}
\begin{center}
\begin{tabular}{|l|c|}
\hline
\multicolumn{2}{|c|}{}                                                         \\
\multicolumn{2}{|c|}{User-Specified Input}                                     \\
\multicolumn{2}{|c|}{}                                                         \\ \hline
                            &                                                  \\
\rb{Parameter}              &  \rb{Value}                                      \\ \hline \hline
$\dot Q$ (kW)               &  516                                             \\ \hline
$\dot m$ (kg/s)             &  4.5                                             \\ \hline
$c\sb{p}$ (\si{kJ/(kg.K)})  &  1.0                                             \\ \hline
$L$ (m)                     &  18.3                                            \\ \hline
$W$ (m)                     &  12.2                                            \\ \hline
$H$ (m)                     &  6.1                                             \\ \hline
$k$ (\si{kW/(m.K)})         &  0.00023                                         \\ \hline
$\rho$ (kg/m$^3$)           &  1000                                            \\ \hline
$c$ (\si{kJ/(kg.K)})        &  1.16                                            \\ \hline
$\delta$ (m)                &  0.025                                           \\ \hline
$T_\infty$ (\si{\celsius})  &  15                                              \\ \hline
\multicolumn{2}{c}{}                                                           \\ \hline
\multicolumn{2}{|c|}{}                                                         \\
\multicolumn{2}{|c|}{Calculated Output}                                        \\
\multicolumn{2}{|c|}{}                                                         \\ \hline
                                 &                                             \\
\multicolumn{1}{|c|}{\rb{Time}}  &  \multicolumn{1}{c|}{\rb{HGL Temperature}}  \\
\multicolumn{1}{|c|}{\rb{(s)}}   &  \multicolumn{1}{c|}{\rb{(\si{\celsius})}}  \\ \hline \hline
\multicolumn{1}{|c|}{0}          &  \multicolumn{1}{c|}{15.00}                 \\ \hline
\multicolumn{1}{|c|}{60}         &  \multicolumn{1}{c|}{34.60}                 \\ \hline
\multicolumn{1}{|c|}{120}        &  \multicolumn{1}{c|}{40.88}                 \\ \hline
\multicolumn{1}{|c|}{180}        &  \multicolumn{1}{c|}{45.16}                 \\ \hline
\multicolumn{1}{|c|}{240}        &  \multicolumn{1}{c|}{48.47}                 \\ \hline
\multicolumn{1}{|c|}{300}        &  \multicolumn{1}{c|}{51.17}                 \\ \hline
\multicolumn{1}{|c|}{360}        &  \multicolumn{1}{c|}{53.47}                 \\ \hline
\multicolumn{1}{|c|}{420}        &  \multicolumn{1}{c|}{55.46}                 \\ \hline
\multicolumn{1}{|c|}{480}        &  \multicolumn{1}{c|}{57.23}                 \\ \hline
\multicolumn{1}{|c|}{540}        &  \multicolumn{1}{c|}{58.81}                 \\ \hline
\multicolumn{1}{|c|}{600}        &  \multicolumn{1}{c|}{60.24}                 \\ \hline
\end{tabular}
\end{center}
\end{table}


\clearpage


\subsection*{Validation}

A summary of the comparisons between peak predicted and measured compartment temperatures is shown in Fig.~\ref{HGL Temperature, Forced Ventilation (DB)}.

\begin{figure}[!ht]
\begin{center}
\includegraphics[width=4.0in]{SCRIPT_FIGURES/Scatterplots/HGL_Temperature_DB}
\end{center}
\caption[Summary of HGL temperature predictions for forced ventilation tests (DB)]
{Summary of HGL temperature predictions for forced ventilation tests using the DB method.}
\label{HGL Temperature, Forced Ventilation (DB)}
\end{figure}


\clearpage


\section{No Ventilation (Beyler)}
\label{sec:Beyler}

\subsection*{Description}

For a compartment with no ventilation (closed doors) and constant HRR, the correlation of Beyler~\cite{SFPE:Walton} predicts that the HGL temperature rise, $\Delta T\sb{g}$~(\si{\celsius}), is given by
\be
\Delta T\sb{g} = \frac{2 K_2}{K_1^2} (K_1 \sqrt{t} - 1 + e^{-K_1 \sqrt{t}})
\label{eq:Beyler}
\ee
where $t$ is the exposure time~(\si{\second}). $K_1$ is given by
\be
K_1 = \frac{2(0.4\sqrt{k \rho c}) A\sb{T}}{m c\sb{p}}
\label{eq:Beyler_K1}
\ee
where $k$ is the thermal conductivity of the interior lining~(\si{kW/(m.K)}), $\rho$ is the density of the interior lining~(\si{kg/m^3}), $c$ is the specific heat of the interior lining~(\si{kJ/(kg.K)}), $A\sb{T}$ the total area of compartment enclosing surfaces~(\si{m^2}), $m$ is the mass of gas in the compartment~(\si{kg}), and $c\sb{p}$ is the specific heat of air~(\si{kJ/(kg.K)}). $K_2$ is given by
\be
K_2 = \frac{\dot Q}{m c\sb{p}}
\label{eq:Beyler_K2}
\ee
where $\dot Q$ is the HRR of the fire~(\si{kW}).


\clearpage


\subsection*{Verification}

This example case is based on Test 1 from the Lawrence-Livermore National Laboratory (LLNL)~\cite{Foote:LLNL1986} series. This test involved a compartment with an open door, a methane burner, and no ventilation.

\begin{table}[!ht]
\caption[Verification case, HGL temperature, no ventilation]
{Verification case, HGL temperature, no ventilation.}
\begin{center}
\begin{tabular}{|l|c|}
\hline
\multicolumn{2}{|c|}{}                                                         \\
\multicolumn{2}{|c|}{User-Specified Input}                                     \\
\multicolumn{2}{|c|}{}                                                         \\ \hline
                            &                                                  \\
\rb{Parameter}              &  \rb{Value}                                      \\ \hline \hline
$\dot Q$ (kW)               &  200                                             \\ \hline
$L$ (m)                     &  6.0                                             \\ \hline
$W$ (m)                     &  4.0                                             \\ \hline
$H$ (m)                     &  4.5                                             \\ \hline
$c\sb{p}$ (\si{kJ/(kg.K)})  &  1.0                                             \\ \hline
$k$ (\si{kW/(m.K)})         &  0.000463                                        \\ \hline
$\rho$ (kg/m$^3$)           &  1607                                            \\ \hline
$c$ (\si{kJ/(kg.K)})        &  1.0                                             \\ \hline
$T_\infty$ (\si{\celsius})  &  23                                              \\ \hline
\multicolumn{2}{c}{}                                                           \\ \hline
\multicolumn{2}{|c|}{}                                                         \\
\multicolumn{2}{|c|}{Calculated Output}                                        \\
\multicolumn{2}{|c|}{}                                                         \\ \hline
                                 &                                             \\
\multicolumn{1}{|c|}{\rb{Time}}  &  \multicolumn{1}{c|}{\rb{HGL Temperature}}  \\
\multicolumn{1}{|c|}{\rb{(s)}}   &  \multicolumn{1}{c|}{\rb{(\si{\celsius})}}  \\ \hline \hline
\multicolumn{1}{|c|}{0}          &  \multicolumn{1}{c|}{23.00}                 \\ \hline
\multicolumn{1}{|c|}{100}        &  \multicolumn{1}{c|}{59.33}                 \\ \hline
\multicolumn{1}{|c|}{200}        &  \multicolumn{1}{c|}{76.72}                 \\ \hline
\multicolumn{1}{|c|}{300}        &  \multicolumn{1}{c|}{90.07}                 \\ \hline
\multicolumn{1}{|c|}{400}        &  \multicolumn{1}{c|}{101.33}                \\ \hline
\multicolumn{1}{|c|}{500}        &  \multicolumn{1}{c|}{111.24}                \\ \hline
\end{tabular}
\end{center}
\end{table}


\clearpage


\subsection*{Validation}

A summary of the comparisons between peak predicted and measured compartment temperatures is shown in Fig.~\ref{HGL Temperature, No Ventilation (Beyler)}.

\begin{figure}[!ht]
\begin{center}
\begin{tabular}{l}
\includegraphics[width=4.0in]{SCRIPT_FIGURES/Scatterplots/HGL_Temperature_Beyler}
\end{tabular}
\end{center}
\caption[Summary of HGL temperature predictions for no ventilation tests (Beyler)]
{Summary of HGL temperature predictions for no ventilation tests using the Beyler method.}
\label{HGL Temperature, No Ventilation (Beyler)}
\end{figure}
